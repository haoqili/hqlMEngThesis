%% This is an example first chapter.  You should put chapter/appendix that you
%% write into a separate file, and add a line \include{yourfilename} to
%% main.tex, where `yourfilename.tex' is the name of the chapter/appendix file.
%% You can process specific files by typing their names in at the 
%% \files
%% prompt when you run the file main.tex through LaTeX.
\chapter{Discussion and Conclusion}

We discussed the CameraDP photo app that uses a distributed ad-hoc network abstraction to carry out user's requests and compared its success rate and latency times to an identical app, CameraCL, that relies purely on the 3G or 4G cellular network. In general the CameraDP app had much lower latencies than the CameraCL app but the success rates were not favorable when used outdoors.

The outdoor experiments were severely limited by the small region sizes. Currently high-power wireless antennas on smart phones are not designed to collaborate with phones in the vicinity, so the Wifi range on phones are too small to be useful for CameraDP.

The promising results of the indoor experiment shed light on how much a distributed ad-hoc app can improve the latency on all the phones. The static indoor experiments (Experiments 5-6) had near 100\% successes, much higher than that of the the mobile indoor experiment (Experiment 4). The range of Wifi on phones could not have have had an influence because the regions sizes and layout were the same in all three indoors experiments. In fact, the static experiments used 6 regions whereas Experiment 4 mainly used only 4 regions. The static experiments were run by only 2 people, so at most 2 requests were carried out simultaneously. Whereas Experiment 4 had 10 volunteers, making many more requests at any given time. This could lead to collision problems and the exposed node problem not yet adapted by the IEEE 802.11 physical layer protocol. The current IEEE 802.11 protocol is designed for at most one moving device, e.g. a stationary hotspot and a moving phone, not for 10 moving phones with ad-hoc Wifi.

With improvements in the IEEE.11 physical layer protocol, the MAC layer collision detection, and the range of phone Wifi range, we hope to see more distributed ad-hoc apps. A increase in the density of smart phones corresponds to a latency decrease on ad-hoc wireless networks, but an unfavorable increase on cellular networks. As smart phones become ubiquitous, it is logical for phones to migrate into distributed ad-hoc network settings.