%% This is an example first chapter.  You should put chapter/appendix that you
%% write into a separate file, and add a line \include{yourfilename} to
%% main.tex, where `yourfilename.tex' is the name of the chapter/appendix file.
%% You can process specific files by typing their names in at the 
%% \files=
%% prompt when you run the file main.tex through LaTeX.
\chapter{User Interface and Functionality of both Camera Apps}

From the user's point of view, CameraDP and CameraCL are identical. Both apps allow users to share photos among themselves using their Android phones.  The users can take new photos on their phones and request to see the latest photos taken by other phones. 

For DIPLOMA But unlike traditional photo sharing apps where each phone functions individually, our experiment assigns the phones into different regions based on their GPS locations and a region’s phones collectively save their photos.  This implies: a) a new photo is saved on its phone’s region, not the phone itself and b) a phone can only request the newest photo of a region, not of another individual phone. 
 
In our experiment, six square consecutive regions (0 to 5) about 30 meters wide were created along a stretch of busy road.  Ideally, there is no limit to the number of regions as long as a phone can only belong to one region at any time. The size of the region should be as big as possible, but still small enough that phones from any points in the region are within WIFI ranges from each other. The users walked around and pressed buttons on the apps to take new photos or request photos from different regions. With two apps on two different phones, the users synchronized the button presses so that sequences of events for the two app types are similar.

The two different apps are: DiplomaCamera and CloudCamera. Even though their UI are identical, DiplomaCamera handles the pictures using DIPLOMA while CloudCamera is the control of the experiment, handling all requests with the cloud.

======
* Use CameraSurfaceView
   The reason for this change was because the *intent/sd card solution only works on Nexus S phones, not Galaxy Notes* phones. In Galaxy Notes, the Mux is killed and restarted before and after the camera intent (because StatusActivity is paused), causing a "Cannot open socketAddress already in use" error: {\url{https://github.com/haoqili/Android_DIPLOMA_CAMERA/blob/8de606e548b4854807ea91a4822b7638a250843c/logcats/galaxy_note_camera_crash.txt#L470}}
    CameraSurfaceView fixes this problem because StatusActivity never has to be paused when taking a photo. Since it works on both types of phones, we used this solution for both.

=========

timeouts
* bug of timeouts
    - *N.B.* Moved code order on leaders to have better completion rates for TAKEs. Before, the buttons are enabled after the leader do all the packet processing, during which time the timeout period could have been reached and counted the completed request as a timedout. Now, enabling the buttons is the first thing the phone does after it receives a packet.
        "It was at the very last step because I thought the photo processing should be done first. But I guess the photo processing stuff is either fast enough or it doesn't really conflict with pressing buttons, because I have not seen a failure after" this change
        {\url{https://github.com/haoqili/Android_DIPLOMA_CAMERA/commit/6164e1ce1699b87bf5a21c724130af9a6d807a19}}
