%% This is an example first chapter.  You should put chapter/appendix that you
%% write into a separate file, and add a line \include{yourfilename} to
%% main.tex, where `yourfilename.tex' is the name of the chapter/appendix file.
%% You can process specific files by typing their names in at the 
%% \files=
%% prompt when you run the file main.tex through LaTeX.
\chapter{Background on DIPLOMA}

The DIstributed Programming Layer Over Mobile Agents (DIPLOMA) \cite{diploma} programming abstraction is only used by CameraDP. In order for the experimental section to make sense, a few aspects of DIPLOMA need to be addressed. 

In DIPLOMA, the phones are assigned into regions \cite{virtualnode} based on their GPS location. (To isolate errors in GPS during debugging, we also allow users to set regions manually.) Given a geographical area of interest, we section the entire space into rectangular regions of equal size and shape. All the phones in the same region act as a single memory unit in DIPLOMA. Theoretically there is no limit to the number of regions, but it is limited by the number of participating phones and the mobility of the phones. Ideally the length of time when regions are empty should be minimized. In addition, the number of phones in each region should be kept at a constant. Since phones are mobile, if a phone walks out of a region it is assigned to a neighboring region and all phones should be assigned to a region. The region width should be big enough so that every phone in the region can broadcast to and hear from every other phone in the region and its neighboring regions.

At any time, a phone is assigned to a state. The states to know for this experiment are LEADER, NONLEADER, and JOIN. The state of a phone may change due to different circumstances. Inside each region one of the phones must be designated to be the LEADER of the region. All the other phones in the regions are NONLEADERS. If a new phone comes into the region from another region or because it's turned on inside the region, it will try to JOIN the region through an exchange with the LEADER. LEADERS inform the NONLEADERS of their continued existence by broadcasting periodic {\it I'm alive} heartbeat packets. The LEADER in the region saves the newest photo data of all phones in its region. The LEADER is also responsible to communicating with other LEADERS to retrieve and relay a remote region's photo to a NONLEADER in the region. This LEADER-to-LEADER communication is a {\it multi-hop} transaction because at any step, neighboring LEADERS relay the request, so the request moves from LEADER to LEADER until the destination LEADER is reached.

In addition to these these ad-hoc Wifi requests, LEADERs have the option to communicate with a cloud server via the cellular 3G or 4G network, just like in CameraCL, but with fewer cloud accesses. In DIPLOMA, the cloud server acts like a last resort for keeping a region's state consistent. For example, if the LEADER phone leaves the old region to go to a neighboring region, the LEADER chooses a potential new LEADER randomly among the NONLEADERs of the region. Normally this LEADER candidate sends an ack to the old LEADER to get the state of the region. However if the old LEADER never hears back from the LEADER candidate or if the old LEADER knows that it was the only phone in the region, then in these cases there are no phones that the old LEADER can pass the state of the region on to. So the old LEADER uploads the region's state to the cloud server, much like what happens for every request in CameraCL. Whenever a new LEADER is formed in a region, under any circumstance, it makes a request to the cloud server for the permission to become the new LEADER. This way, the cloud server can prevent double LEADERs from forming in the same region. If the new LEADER is approved by the cloud, it also receives from the cloud any old states of the region, so that data in the regions can be remain consistent.

LEADERs also send heartbeats to the cloud server to announce that they are still alive. In case that the old LEADER phone is turned off or crashes, the cloud server can quickly grant leadership to another phone when it detects multiple skipped cloud heartbeats from the old LEADER. Generally DIPLOMA should try to have as few cloud accesses as possible, to reduce latency. However, the cloud heartbeats should not be made too infrequently that potential new LEADERs of a region with a dead LEADER have to wait for a very long time, during which they would repeat leadership cloud requests. The length of the heartbeat period should be customized according to the app and the phones.
%%omit cloud rejecting leadership (timeouts) for a while after leader crashes, and leader HANDINGOFF, they do not directly affect Camera app%%

The width of DIPLOMA regions are of great concern to CameraDP as well as CameraCL, because both apps share the same region assignment. As mentioned before, each region has to be small enough so that every phone in the region are in range with each other, but also large enough to ensure sufficient density in one region. Recall that LEADERs of neighboring regions must be able to communicate with each other. If there is any one region that is missing a LEADER or has an out-of-range LEADER along a linear path of multi-hop, then the DIPLOMA request is broken and the request cannot be completed successfully. In other words, if there is a chain of regions, all regions must have a LEADER and that LEADER must be in range with its neighboring LEADERs. This implies that we must make the region size small enough that phones anywhere inside two adjacent regions, not just one, could hear each other.  How wide should a regions to be? If the region widths are set exactly as the limiting range of the phones (20 meters in our case), the only way a DIPLOMA multi-hop would work is if the LEADERs are exactly 20 meters from each other. If one of the LEADERs just moves a little bit, it will fall out of range of the farther neighboring LEADER and thus breaking DIPLOMA multi-hop requests.

Even though technically the region width should be half of the phone range, with the GPS inaccuracy of the phones, setting the region width to 10 meters is not ideal. If the regions are too small, and the phone's innate GPS inaccuracies varies a lot, we could end up with incorrect region allocations. In the worst case, region monotonicity may be broken, e.g. a phone could be erroneously assigned to region 2, between a region 3 and region 4 phone. Without region monotonicity, DIPLOMA multi-hop routing would not function since it uses Greedy Perimeter Stateless Routing \cite{gpsr}. For instance, we found in Experiment 3 that setting the region width down from 20 meters in width to 10 meters in width did not improve the rate of success at all.

\begin{figure}[htb]
\begin{center}
\includegraphics[width=14cm]{hysteresis-reg.png}
\caption{Hysteresis zones are the dashed areas, where a phone's region is frozen. The red and blue colors correspond to two different phones that have different GPS offsets, which lead to diffferent region boundaries (thick vertical lines). The white areas are were both phones are definitely going to be assigned to that region. This diagram shows only 2 phones' hysteresis zones. With 10 phones the total hysteresis zones are going to be even larger.}
\label{fig:hysteresis-png}
\end{center}
\end{figure}

In the first two experiments where the region widths were 52 meters, we reserved 10 meters around each region boundary as the hysteresis buffer zone where the region on a phone cannot be changed. This was to avoid phones near region boundaries flickering too quickly due to impreciseness of the GPS, thus having to JOIN constantly. Hysteresis buffer zones worked well for large regions. However, having hysteresis at smaller regions, such as 20 meters instead of 52 meters, created more harm than benefits. Since the GPS is imprecise and in addition there are different innate GPS offsets on each phone, the combined hysteresis buffer zone region from all the phones would take up the majority area of a region, i.e. there was less than 40\% of a region where all the phones agreed on what the region would be (Figure \ref{fig:hysteresis-png}). In these relatively large hysteresis zones, we often observed phones side-by-side getting assigned to different regions, sometimes even two regions apart, which would break region monotonicity. So from Experiment 3 onwards, we stopped using hysteresis. Luckily, the impreciseness of the GPS was found to be not a concern. (Even though we added a hysteresis selector button that could change the width of the hysteresis zone during the experiment, we always just used the default of 0.)