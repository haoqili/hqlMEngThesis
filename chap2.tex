%% This is an example first chapter.  You should put chapter/appendix that you
%% write into a separate file, and add a line \include{yourfilename} to
%% main.tex, where `yourfilename.tex' is the name of the chapter/appendix file.
%% You can process specific files by typing their names in at the 
%% \files=
%% prompt when you run the file main.tex through LaTeX.
\chapter{Background on DIPLOMA}

The DIstributed Programming Layer Over Mobile Agents (DIPLOMA) programming abstraction only runs on CameraDP. In order for the experimental section to make sense, a few aspects of DIPLOMA need to be addressed. Before doing so, keep in mind that a paper detailing DIPLOMA is, as of May 2012, in the process of being submitted to a conference. So we won't be able to reference it here, but feel free to look it up online for more details on DIPLOMA.

In DIPLOMA, the phones are assigned into regions based on their GPS location. Given a map of the entire area of interest, the map should be sectioned off into invisible regions so that the phones inside the same region are close to each other in physical distance. Theoretically there is no limit to the number of regions you can make, but it is limited by the number of participating phones. Ideally the length of time of regions being empty should be minimized. So more phones correspond to more regions. Since phones are mobile, if a phone walks out of a region it is assigned to a neighboring region. There should be no overlapping of regions or unassigned phones. Every phone in the region should be technically able to hear from every other phone in the region through ad-hoc Wifi. 

Inside each region one of the phones is designated to be the LEADER of the region. All the other phones in the regions are NONLEADERS. If a new phone comes into the region from another region or because it's turned on inside the region, it will try to JOIN the region through an exchange with the LEADER. LEADERS inform the NONLEADERS of its continued existence by broadcasting periodic {\it I'm alive} heartbeat packets. The LEADER in the region saves photo data that the NONLEADERS (and itself) take. The LEADER is also responsible to communicating with other LEADERS to retrieve and relay a remote region's photo to a NONLEADER in the region. This LEADER-to-LEADER communication is called {\it multi-hop} because at any step, neighboring LEADERS relay the request, so the request moves from LEADER to LEADER until the destination LEADER is reached.

In addition to these these ad-hoc Wifi requests, LEADERs have the right and privilege to communicate with a cloud server via the cellular 3G or 4G network, just like in CameraCL, but with fewer cloud accesses. In CameraDP, the cloud server acts like a last resort for keeping a region's state consistent. For example, if the LEADER phone leaves the old region to go to a neighboring region, the NONLEADERs of the old region will detect that by the missed LEADER to NONLEADER heartbeat packets. Within a few seconds, the NONLEADERs will randomly choose a new LEADER among themselves. This potential new leader will try to send a packet to the old leader, from which the old leader will give the new leader its states. However if the old LEADER never hears back from the newly elected LEADER or if the old LEADER knows that there are no other phones in the region, then in this case there are no phones that the old LEADER can pass on the state of the region to. So the old LEADER uploads the region's state to the cloud server (much like what happens for every request in CameraCL). Whenever a new leader is formed in a region, under any circumstance, a cloud request is made to ask the cloud server to give it permission to become the new leader (preventing double leaders in a region). If the new leader is approved by the cloud, it also receives from the leader any old states of the region, so that data for that region can remain consistent.

LEADERs also send heartbeats to the cloud server to announce that they are still viable. This way in case that the old LEADER phone turned off or crashed, the cloud server can quickly grant leadership to another phone when it detects multiple skipped cloud heartbeats. Generally CamerDP should try to have as few cloud accesses as possible, to reduce latency. However, the cloud heartbeats should not be made too infrequently that potential new LEADERs of a region with a dead LEADER have to wait for a very long time, during which they would repeated leadership cloud requests. The length of the heartbeat period should be customized according to the app and the phones.
%%omit cloud rejecting leadership (timeouts) for a while after leader crashes, and leader HANDINGOFF, they do not directly affect Camera app%%

Another aspect of DIPLOMA that is important to the CloudDP app is its regions. As mentioned before, each region have to small enough so that every phone in the region are in range with each other, but also big so that more phones can be contained in one region. Recall that leaders of neighboring regions must be able to communicate with each other. If there is any one region that is missing a LEADER or have an out-of-range LEADER along a linear path of multi-hop, then the DIPLOMA request is broken and the request cannot be completed successfully. In other words, if there is a chain of regions, all regions must have a leader an that leader must be in range with its neighboring leaders. This implies that we must make the region size small enough that two phones anywhere inside two adjacent regions, not one, could hear each other.  But how wide should a regions to be? If the region widths are set exactly as the limiting range of the phones (20 meters in our case), the only way a DIPLOMA multi-hop would work is if the leaders are exactly 20 meters from each other. If one leaders just moves a little bit, it will fall out of range of the farther neighboring leader and thus breaking DIPLOMA multi-hops.

Even though technically the region width should be half of the phone range, with the GPS inaccuracy of the phones, setting the region width to 10 meters is not ideal. If the regions are too small, and the phone's innate GPS inaccuracies varies a lot, we could end up with insensible region allocations. In the worst case, region monotonicity may be broken, e.g. a phone could be erroneously assigned to region 2, between a region 3 and region 4 phone. Without region monotonicity, DIPLOMA multi-hop would not be concrete. (We found in Experiment 3 that setting the region width down from 20 meters in width to 10 meters in width did not improve the rate of success at all.)

Hysteresis was used around region boundaries in the first two experiments but was not used due to its complications in smaller regions of 20 meters or fewer in Experiment 3 onwards. In the first two experiments where the region widths were 52 meters, we used reserved 10 meters around each region boundary as the hysteresis buffer zone where the region on a phone cannot be changed. At that time we were worried that the GPS would be too imprecise that phones standing near region boundaries might flicker between the two regions very quickly, constantly having to JOIN. Hysteresis buffer zones worked well for large regions. However, having hysteresis at smaller regions, such as 20 meters instead of 52 meters, brought more confusion than benefits. At a majority of points on the grounds of the experiment, there would be some phones assigned to inside a hysteresis region, i.e. there would be about five meters where all phones agree on a region. This relatively large area of hysteresis, caused by GPS inaccuracies and different internal GPS offsets on each phone, led to phones next to each other getting assigned to different regions, sometimes even two regions apart, which would break region monotonicity. Before Experiment 3, we added a hysteresis selector button that would choose different width for hysteresis, but we just set the hysteresis to 0.