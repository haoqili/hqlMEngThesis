%% This is an example first chapter.  You should put chapter/appendix that you
%% write into a separate file, and add a line \include{yourfilename} to
%% main.tex, where `yourfilename.tex' is the name of the chapter/appendix file.
%% You can process specific files by typing their names in at the 
%% \files=
%% prompt when you run the file main.tex through LaTeX.
\chapter{User Interface and Functionality Common to Both Camera Apps}

CameraDP and CameraCL, unlike traditional photo sharing apps where each phone functions individually, assigns the phones into different regions based on their GPS locations and a region'€™s leader collectively saves the newest photos for all the phones in the region (it's easy to change the code to save more than one phones). This implies: a) a new photo is saved on its phone'€™s region, not the phone itself and b) a phone can only request the newest photo of a region, not of another individual phone. 

From the user's point of view, CameraDP and CameraCL are identical. Both apps allow users to share photos among themselves using their Android phones.  The users can take new photos on their phones, by pressing the "Take Photo" button and request to see the latest photos taken by other phones by pressing the "Get X Photo" button where "X" correspond to the desired region number.  The "Take Photo" button press triggers a TAKE request in CameraDP and the "Get X Photo" button press triggers a GET request. These are the two main requests crucial to our experiments.

The rest of the UI are add-ons to help debugging in the process. Log messages are displayed in the middle. Success rates of TAKEs and GETs are displayed on the bottom of the screen, along with request latency information. The textfield is for setting a new region width. Remember when changing the region width on one phone, all the other phones involved must have the region width changed as well to keep the region assignments consistent among all phones. 

The last button is a switch for hysteresis, allowing you to pick different percentages of the region width you want to be applied to be the hysteresis buffer region. If hysteresis is set, the region of the phone cannot be changed inside the hysteresis region, the few meters (based on the hysteresis percentage chosen) around the boundaries of the regions. Hysteresis was set to 0 after Experiment 2 due to its complications discussed in the previous chapter.




======
* Use CameraSurfaceView bug discussion?
   The reason for this change was because the *intent/sd card solution only works on Nexus S phones, not Galaxy Notes* phones. In Galaxy Notes, the Mux is killed and restarted before and after the camera intent (because StatusActivity is paused), causing a "Cannot open socketAddress already in use" error: {\url{https://github.com/haoqili/Android_DIPLOMA_CAMERA/blob/8de606e548b4854807ea91a4822b7638a250843c/logcats/galaxy_note_camera_crash.txt#L470}}
    CameraSurfaceView fixes this problem because StatusActivity never has to be paused when taking a photo. Since it works on both types of phones, we used this solution for both.

=========

timeouts
* bug of timeouts discussion?
    - *N.B.* Moved code order on leaders to have better completion rates for TAKEs. Before, the buttons are enabled after the leader do all the packet processing, during which time the timeout period could have been reached and counted the completed request as a timedout. Now, enabling the buttons is the first thing the phone does after it receives a packet.
        "It was at the very last step because I thought the photo processing should be done first. But I guess the photo processing stuff is either fast enough or it doesn't really conflict with pressing buttons, because I have not seen a failure after" this change
        {\url{https://github.com/haoqili/Android_DIPLOMA_CAMERA/commit/6164e1ce1699b87bf5a21c724130af9a6d807a19}}

