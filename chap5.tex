%% This is an example first chapter.  You should put chapter/appendix that you
%% write into a separate file, and add a line \include{yourfilename} to
%% main.tex, where `yourfilename.tex' is the name of the chapter/appendix file.
%% You can process specific files by typing their names in at the 
%% \files=
%% prompt when you run the file main.tex through LaTeX.
\chapter{CameraCL Android Application}

In CameraCL, every request is sent to the cloud server. The cloud server keeps a dictionary linking each region to its newest photo.  CameraCL only has one important file: CameraCL.java that is analogous to CameraDP's StatusActivity.java, but instead of sending UDP packets, CameraCL sends HTTP post requests. Latency is calculated from the difference of the time stamps surrounding the line that executes the http request.

The code that assigns regions in CameraCL is identical to the code that assigns regions in CameraDP (see Chapter 3). Even though there are regions in CameraCL, all the phones a region are treated equally, i.e. there are no LEADERs or NONLEADERs.

The cloud server returns a status for every request. For TAKE requests, this status indicates if the photo was saved successfully. For GET Requests, a status of failure does not distinguish between a null region or a region with phones but not yet any photos, because CameraCL's cloud server only knows the existence of a region from the region's first TAKE request. Since CameraCL phones do not have to communicate with the cloud server when changing regions like CameraDP's taking or releasing leadership, it is possible to GET photos from a region successfully even if all phones have left that region.