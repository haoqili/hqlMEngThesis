%% This is an example first chapter.  You should put chapter/appendix that you
%% write into a separate file, and add a line \include{yourfilename} to
%% main.tex, where `yourfilename.tex' is the name of the chapter/appendix file.
%% You can process specific files by typing their names in at the 
%% \files=
%% prompt when you run the file main.tex through LaTeX.
\chapter{Introduction and Motivation}

Smart phones use the conventional client-server programming model to query a centralized cloud server through the cellular network for data computation, processing, and storage. However with the number of smart phone users and data-intensive apps on the rise, cellular networks are often overloaded, causing increased latencies on the phones and frustration among the users. A solution is to move to a distributed programming model where phones collaborate among themselves through the short-range ad-hoc Wifi network.  Assuming reliable and strong ad-hoc Wifi conditions, requests on Wifi to nearby phones should be faster on average than requests on 3G (HSPA) and 4G (LTE) network to the cloud, improving the user experience. 

This is an area of active research and we make an app that utilizes a recently built consistent shared-memory system over ad-hoc Wifi, named DIstributed Programming Layer Over Mobile Agents (DIPLOMA) \cite{diploma}, to test the feasibility of the popular Panoramio \cite{Panoramio} app on a distributed setting. Panoramio is a popular location-based photo-sharing that links photos to their GPS coordinates. Users can new upload photos of a location and retrieve photos of different locations.

We created a stripped-down version of Paranamio that only has two functions: taking new photos and getting other photos. In order to quantify the advantage of ad-hoc WiFi, we built two functionally identical apps: CameraDP and CameraCL. CameraDP uses DIPLOMA in the background while CameraCL is the control app where each request is independently sent to the cloud through a 3G or 4G connection.

We conducted six experiments while continuing to improve the codebase. Two types of Android phones were used: Nexus S \cite{nexus} and Galaxy Note \cite{galaxy}. During the experiments, users pressed buttons to take or get pictures. Some experiments were conducted with volunteers walking around outdoors with the phones while pressing the buttons, simulating real-life situations. Other experiments were done in a static setting indoors, where the phones do not move. We logged and analyzed the number of success of requests and the latency of the responses.

We will first introduce DIPLOMA and discuss a few of its details relevant to our experiments in Chapter 2. Then in Chapter 3 we will go over the user interface, which is common to both CameraDP and CameraCL. The parts that are unique to each app are in Chapters 4 and 5 respectively. In Chapter 6 we then describe in detail each of the six experiments and what improvements were made after each experiment. Finally we conclude in Chapter 7.