%% This is an example first chapter.  You should put chapter/appendix that you
%% write into a separate file, and add a line \include{yourfilename} to
%% main.tex, where `yourfilename.tex' is the name of the chapter/appendix file.
%% You can process specific files by typing their names in at the 
%% \files=
%% prompt when you run the file main.tex through LaTeX.
\chapter{Introduction and Motivation}

Smart phones heavily rely on the Cloud to carry out extensive computations or get access to abundant storage.  Frequent communication with the Cloud, especially in an area dense with smartphones, can cause problems such as decreased bandwidth, decreased response time, and reduced battery life, among which the increased latency time causes an immediate frustration to the users.

A theoretical solution to this problem is to set the phones in a distributed shared-memory network. Given a good enough WiFi condition, requests to nearby phones should be faster than requests to the cloud, increasing user experimence on the phone. This is an area of active research and we utilize a recently built consistent shared-memory system, DIPLOMA, to test the feasibility of a popular photo-sharing app, Paranomio, on a distributed memory system relying mostly on the ad-hoc WiFi.

We created a stripped-down version of Paranamio that only have two functions: {\it take} new photos and {\it get} other's photos. In order to quantify the advantage of ad-hoc WiFi, we built two functionally identical apps: CameraDP and CameraCL. CameraDP uses DIPLOMA in the background while CameraCL is the control case where each request is independently sent to the cloud. one using DIPLOMA (CameraDP) and the other using 3G or 4G.
