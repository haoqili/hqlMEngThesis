%% This is an example first chapter.  You should put chapter/appendix that you
%% write into a separate file, and add a line \include{yourfilename} to
%% main.tex, where `yourfilename.tex' is the name of the chapter/appendix file.
%% You can process specific files by typing their names in at the 
%% \files=
%% prompt when you run the file main.tex through LaTeX.
\chapter{Introduction and Motivation}

Smart phones rely heavily on the Cloud to carry out extensive computations or get access to abundant storage.  Frequent communication with the Cloud via the 3G (HSPA) or 4G (LTE) cellular networks, especially in an area dense with smartphones, can cause an increased latency time, to the immediate frustration of the users. A solution to this problem is to set the phones in a distributed shared-memory network. Given reliable and strong ad-hoc Wifi conditions, requests to nearby phones should be faster on average than 3G or 4G requests to the cloud, improving the user experience. 

This is an area of active research and we utilize a recently built consistent shared-memory system over ad-hoc Wifi, DIPLOMA, to test the feasibility of a popular photo-sharing app, Paranomio, on a distributed memory system relying mostly on the ad-hoc WiFi.

We created a stripped-down version of Paranamio that only have two functions: TAKE new photos and GET other photos. In order to quantify the advantage of ad-hoc WiFi, we built two functionally identical apps: CameraDP and CameraCL. CameraDP uses DIPLOMA in the background while CameraCL is the control case where each request is independently sent to the cloud through a 3G or 4G connection.